\documentclass[letterpaper]{article}
\usepackage{fullpage}
\usepackage{amsmath}
\usepackage{graphicx}
\usepackage{setspace}
\usepackage{booktabs}
\usepackage{multirow}
\usepackage{cite}
\usepackage[labelfont=bf]{caption}
\usepackage{titlesec}
%\usepackage[nomarkers,tablesfirst]{endfloat}
\titleformat{\section}[block]{\large\filcenter\bfseries}{\thesection}{1em}{}

% abstract
\usepackage[runin]{abstract}
\renewcommand{\abstractname}{}
\renewcommand{\abstracttextfont}{\normalfont\small\bfseries}

% reformat section/subsection headings
\renewcommand{\thesection}{\Roman{section}.}
\renewcommand{\thesubsection}{\Alph{subsection}.}
\renewcommand{\thesubsubsection}{\arabic{subsubsection}}

%Manuscript style
\linespread{2}
%\usepackage[nomarkers]{endfloat}

\setcounter{secnumdepth}{5}

 \title{\bf MAAWM SIMULATION SOFTWARE DOCUMENTATION}

 \author{
  Carlos Montalvo (cmontalvo@southalabama.com) \thanks{Assistant
    Professor,Department of Mechanical Engineering} \\
{\normalsize\itshape
    University of South Alabama, Mobile, AL 36688}\\}
\date{}
 % Data used by 'handcarry' option if invoked
 %% \AIAApapernumber{2012}
 %% \AIAAconference{Conference Name, Date, and Location}
 %% \AIAAcopyright{\AIAAcopyrightD{2012}}

 % Define commands to assure consistent treatment throughout document
\newcommand{\eqnref}[1]{(\ref{#1})}
 \newcommand{\class}[1]{\texttt{#1}}
 \newcommand{\package}[1]{\texttt{#1}}
 \newcommand{\file}[1]{\texttt{#1}}
 \newcommand{\BibTeX}{\textsc{Bib}\TeX}


\begin{document}
\maketitle
\begin{abstract}

The Multi-Agent Atmospheric Windmapper (MAAWM) simulation sofware is
capable of simulating N aircraft using a full 6DOF aircraft model
which uses a taylor series expansion about a trim point for
aerodynamics. Each aircraft samples the atmosphere at a specified
sample rate and writes the wind data to a file. This file is read by a
radial basis surrogate model routine to fit the data sampled. The
following report documents all input files necessary 
to run the code.

\end{abstract}

\newpage

\tableofcontents

\newpage

\section{Source Code}

All source code is located in 'source/'. The source
code is written in the FORTRAN 90 standard using modules to organize
variables. At the top of the code starts off the module
MAAWMDATATYPES. This data type has numerous subtypes ranging from the
AIRCRAFTSTRUCTURE all the way to a CONTROLSYSTEMSTRUCTURE. Each datatype
has attributes associated with it. For example the MAAWMSTRUCTURE has
an ATMOSPHERESTRUCTURE called ATM. THE ATMOSPHERESTRUCTURE
has many variables, one being VXWIND. Throughout the code
the MAAWMSTRUCTURE is initialized with variable name 'T'. Thus if the
variable VXWIND is needed the coder can 
simply enter 'T\%ATM\%VXWIND'. This is the syntax of data structures
throughout the code. ATM is a structure of datatype 'T' and VXWIND is
an attribute of the ATM structure. The entire code is setup like
this. 

The main PROGRAM MAAWM starts by reading the argument given
to the main code and it then reads it and loads all the necessary
input files. The code then begins to load all input
files. Each subroutine has a load (1), echo (2) and compute (3). For
example, the subroutine ATMOSPHERE can be called with flag 1
for load. The following lines load all of the necessary attributes for
the simulation. After some write(25,*) commands which write to the log
file, the subroutines are called again except with the echo flag
tripped. Here all attributes are echoed to the log
file. Finally, the SIMULATION subroutine is called with the
compute flag tripped. The SIMULATION subroutine kicks off an RK4 by
calling a subroutine SYSTEMDERIVATIVES to compute the state
derivatives. SYSTEMDERIVATIVES contains all subroutine calls to each
module. The CONTROL subroutine is called once per timestep rather than
during each state derivatives function call. Each system has its own
subroutine. The aircraft model is AIRCRAFT. These subroutines will
most likely stay constant however the 
CONTROL subroutine will be edited heavily. 

\section{Input Files}

The input files here use the standard structure of a main file of
files that lists where to point to all the input files for each
module. The sections below detail all the input files used in the
simulation. 

\subsection{IFILES File}

The example file of files given in this bundle is called 'MAAWM.ifiles'
located in 'Input\_Files/'. 

Each file is very self explanatory. For example, 'WRF.ATM' contains
all the parameters for the atmosphere. In addition, 'Decathalon.AC'
contains all the parameters for the aircraft. However, each file is further
discussed in detail in the sections that follow.

\subsection{ATM File}

The atmospheric file defines the atmosphere that the entire system
will fly through. There are two types of atmospheres that are
coded. The first is a constant wind field with an azimuth and
elevation. The other is the standard Weather Research and Forecasting (WRF)
Model plus a Full Field Dryden Turbulence Model. The file structure of
the constant wind field is shown below.

{\singlespace
\begin{tabular}{l l l}
1	& !Atmospheric Wind Type(1=constant,4=WRF Model) \\
1.22566 	& !Density$(kg/m^3)$\\
2.0	& !Windspeed$(m/s)$\\
1.5707 & !Wind direction\_Psi(rad) (0 = tailwind, 3.141 = headwind,sidewind=1.5707)\\
0.0     & !Wind Elevation Theta(rad) (1.5708 thermal)\\
\end{tabular}
}%\endsinglspace
\\

The first line is the type of wind field which in this case is
constant. The second line is the density of the air. The third line is
the magnitude of the wind vector $(V_w)$ which in this case is 2.0
$m/s$. The last two lines are the wind azimuth ($\psi_w$) and
elevation ($\eta_w$). The inertial components of the wind velocity are
then given by the equation below. 

\begin{equation}
\begin{Bmatrix} V_x \\ V_y \\ V_z \end{Bmatrix} = V_w \begin{Bmatrix}
  cos(\psi_w)cos(\eta_w) \\ sin(\psi_w)cos(\eta_w) \\ -sin(\eta_w) \end{Bmatrix}
\end{equation}

Note that the atmospheric windspeed is subtracted from the body
velocity component of each body. For example, the aerodynamic velocity
of the payload is given by the equation below where ${\bf T_{IP}}$ is the
transformation matrix from the inertial frame to the payload frame.

\begin{equation}
\vec{V}_{P_{atm}/I} = \vec{V}_{P/I} - {\bf T_{IP}}^T\vec{V}_{ATM/I}
\end{equation}

The second atmospheric wind field that can be used is the WRF+Dryden
Turbulence Model. The structure of the input file is given below.

{\singlespace
\begin{tabular}{l l l}
4	& !Atmospheric Wind Type(1=constant,4=WRF Model)\\
1.22566 	& !Density$(kg/m^3)$\\
1.0	&!Wind scale ($\sigma_{WRF}$)\\
1.0	& !Turblevel ($\sigma_T$)\\
0	& !Offset Heading(rad) ($\psi_W$) \\
-2.75	& !Vx\_Wave\_speed$(m/s)$ ($V_{xW}$)\\
2.75	& !Vy\_Wave\_speed$(m/s)$ ($V_{yW}$)\\
'WRF\_Wind\_Data/' & !WRF Data location
\end{tabular}
}
\\

Again the first line sets the type of wind field to the WRF model. The
second line sets the density of the air. The next 5 lines set the
scale and offset of the WRF model. The WRF model is a complex
simulation tool that outputs inertial velocity components as a
function of space and time. The TAPAS simulation software requires
that the WRF model be run ahead of time and outputted to text files
for import. The last line dictates the location of the WRF files. This
report will not go into detail on how to obtain the WRF files and will
assume that the files are already present in a working directory. The
TAPAS simulation imports these text files and uses a 3-D interpolation
using x,y,z as independent variables. Although the WRF model is
capable of outputting winds as a function of time the wind field is
assumed to by static to speed computation. However, to avoid limit
cycles from flying through a constant wind field, a scale factor
and offset are used to change the wind field according to the equation
below where the functions WRF and DRYDEN are interpolated during
simulation.

\begin{equation}
\begin{Bmatrix} V_x \\ V_y \\ V_z \end{Bmatrix} = \sigma_{WRF} \begin{Bmatrix}
  WRF_x(x^*,y^*,z) \\ WRF_y(x^*,y^*,z)
  \\ WRF_z(x^*,y^*,z) \end{Bmatrix} + \sigma_{T} \begin{Bmatrix}
  DRYDEN_x(x,y,z) \\ DRYDEN_y(x,y,z) \\ DRYDEN_z(x,y,z) \end{Bmatrix}
\end{equation}

In the equations above $x^*$ and $y^*$ are given by the equations
below.

\begin{equation}
\begin{matrix}
x_R = x cos(\psi_W) + y sin(\psi_W) \\
y_R = -x sin(\psi_W) + z cos(\psi_W) \\
x^* = x_R - V_{xW}(t-t_0)\\
y^* = y_R - V_{yW}(t-t_0)
\end{matrix}
\end{equation}

\subsection{AC File}

The aircraft file is very straight forward. The file starts with flags
to turn the module, dynamics, gravity, aerodynamics, and contact
forces on and off. Then the mass, inertia and area properties are
defined. Finally the aerodynamics are set using a standard
aerodynamic expansion for an aircraft about a trim point. An example
aircraft file can be found in 'Input\_Files/Decathalon.AC' which is an
AC file for a scaled Decathalon aircraft.

\subsection{CS File}

The CS file defines all of the controllers currently coded as well as
the gains associated with each PID controller and the amount of sensor
errors. An example CS file can be found in
'Input\_Files/MaaWM.CS'. 

\subsection{SIM File}

The SIM file defines the time and initial conditions of the system as
well as a few extra flags that can be set. An example SIM file can be
found in 'Input\_Files/MaaWM.SIM'. The first three lines define the
initial and final time of the simulation as well as the timestep. The
output skip parameter defines the number of timesteps the simulation
will skip before the state of the system is outputted to a file. The
4th line turns the control system on and off. The next line sets
whether or not the aircraft will wander aimlessly like a Roomba or
follow the waypoints set in the Waypoints.WAY file.

\subsection{ICS and WAY Files}

The .ICS and .WAY files are generated by MATLAB codes
Create\_Initial\_Conditions.m and Create\_Waypoints.m
respectively. The first code creates random initial conditions for all
aircraft and the second generates a vector of waypoints to
follow. The code is designed to import these files, the aircraft then
have a standard control system set to follow the waypoints in the file.

\section{Output Files}

The code makes use of numerous output files for plotting and
debugging. All files listed below are outputted to the
folder 'Output\_Files/'. 

\subsection{LOG File}

The first and most important output file is the LOG file. The LOG file
contains information on every parameter defined in the input files
above. If the simulation does not run often times an input file is
defined incorrectly. It is possible to consult the log file to see
where something may have gone wrong. 

\subsection{SOUT File}

The SOUT or STATE out file contains all state information as a
function of time with as many data points as defined by the output
skip parameter in the SIM file. The SOUT file has the following
structure: $t,Aircraft States$. The
aircraft states are defined as
$x,y,z,\phi,\theta,\psi,u,v,w,p,q,r$. Each aircraft outputs its own
state vector next to the first. Thus, if two aircraft are in the space
the SOUT file will be 25 columns. The first column will be the time
vector, the next 12 will be the first aircraft states and the next 12
will be the next aircraft states.

\subsection{MOUT File}

The MOUT file is a MISCELLANEOUS output parameter that can be used to
output whatever the user wants. 

\subsection{COUT File}

The COUT file contains useful outputs associated with the control
system. 

\subsection{FOUT File}

The FOUT file is the FORCE output file which contains aerodynamic
loads on the aircraft.

\subsection{EOUT File}

The EOUT file is the ERROR out file which contains the polluted and
unpolluted signals used for feedback.

\subsection{WOUT File}

The WOUT file is the WIND file. All aircraft output Vx,Vy,and Vz to
this file. This file always has 7 columns. The first column is time,
the next three are the x,y,z coordinates of the sample point and the
last three columns are the wind magnitudes along each axis. The first
row is the total number of data points sampled.

\section{Running Simulations}

In order to run the simulation the code must first be compiled. The
ifort or gfortran compiler will suffice. I use a linux machine (Ubuntu
14.04). It is also possible to compile the source code using a
make file. The 'Makefile' I am using is located in the root
directory. With this file present in the root directory I can merely
type the command 'make' into a terminal.

{\it \$ make}

If the code is successfully compiled and no errors are returned a file
name 'Run.exe' should appear in the root directory. At this point the
run command may be used to execute the program. Note that the ifiles
file must be given as a command line argument.

{\it \$ ./Run.exe Input\_Files/MaaWM.ifiles'}

The code may return errors if files have not been found but the error
catching in the code should catch everything you require. The example
simulation setup currently takes about 35 seconds on my machine. 

\section{Using MATLAB or Octave to Run Everything}

I have created a .m script file that handles all of the compiling,
running and plotting. The script file is called 'runMAAWM.m'. This
file contains numerous flags associated with is such as changing the
number of aircraft and recomputing ICS or Waypoints however as it is
setup right now the script will compile the code if necessary and then
plot the trajectories of all aircraft in the space.

{\it $>>$ runMAAWM}

\end{document}

